\documentclass[conference]{IEEEtran}
\IEEEoverridecommandlockouts
% The preceding line is only needed to identify funding in the first footnote. If that is unneeded, please comment it out.
%Template version as of 6/27/2024

\usepackage{float}
\usepackage{cite}
\usepackage{amsmath,amssymb,amsfonts}
\usepackage{algorithmic}
\usepackage{graphicx}
\usepackage{textcomp}
\usepackage{xcolor}
\def\BibTeX{{\rm B\kern-.05em{\sc i\kern-.025em b}\kern-.08em
    T\kern-.1667em\lower.7ex\hbox{E}\kern-.125emX}}
\begin{document}

\title{PuckBot: Predictive Interception and Striking Control for IIWA Air Hockey
}

\author{\IEEEauthorblockN{1\textsuperscript{st} Ryan Quinn}
\IEEEauthorblockA{\textit{CSAIL} \\
\textit{MIT}\\
Cambridge, MA \\
rquinn1@mit.edu}
\and
\IEEEauthorblockN{2\textsuperscript{nd} Tye Phoenix}
\IEEEauthorblockA{\textit{CSAIL} \\
\textit{MIT}\\
Cambridge, MA \\
tphoenix@mit.edu}
\and
\IEEEauthorblockN{3\textsuperscript{rd} Guillermo Mendoza}
\IEEEauthorblockA{\textit{CSAIL} \\
\textit{MIT}\\
Cambridge, MA \\
gdmen@mit.edu}
}


\maketitle

% \begin{abstract}
% This document is a model and instructions for \LaTeX.
% This and the IEEEtran.cls file define the components of your paper [title, text, heads, etc.]. *CRITICAL: Do Not Use Symbols, Special Characters, Footnotes, 
% or Math in Paper Title or Abstract.
% \end{abstract}

% \begin{IEEEkeywords}
% component, formatting, style, styling, insert.
% \end{IEEEkeywords}

\section{Overview}
Air hockey is a fast-paced tabletop game where two players use circular paddles to strike a sliding puck into the opponent’s goal. The puck glides with near-zero friction, bounces off walls, and must be hit both quickly and accurately. Success depends on both offense (striking the puck in a way that pressures the opponent) or defense (positioning the paddle to block or redirect incoming shots). Players earn points by scoring goals, with the winner typically being the first to reach seven. 

Our project aims to create a program in which two robotic IIWA arms simulate a game of air hockey. Both IIWAs will be manipulating air hockey paddles to strike the puck, which is simulated on a virtual low-friction surface. The objective is to maximize both offensive and defensive performance, allowing the robots to mimic human gameplay.

Air hockey is extremely fast-paced and requires precise manipulation. This type of environment is where most robots struggle. Players have milliseconds to decide where to move their paddle so that it intersects the puck in an advantageous way. For example, a shot that is on target (or has a velocity towards the opposing player’s goal) provides an advantage because it requires the opposing player to act defensively. Even small miscalculations can lead to own goals, awkward bounces, or easy opportunities for opponents.

\begin{figure}[h!]
    \centering
    \includegraphics[width=0.75\linewidth]{IMG_3974.jpg}
    \caption{ Trajectory planning for paddles to intersect pucks while they are in motion.}
    \label{fig:placeholder}
\end{figure}

This makes air hockey a compelling benchmark for robotics. It simulates the fast manipulation of objects and requires real-time estimation of the puck’s position and velocity. These core challenges are fundamental to robotics and extend to broader applications in movement (walking/running), sports, and everyday tasks. Producing desirable strikes is difficult and requires the player to adjust the speed, orientation, and timing of their movements simultaneously. By integrating motion planning, perception, and inverse kinematics, we will deliver a reproducible IIWA simulator with real-time prediction and feasible intercept planning.

\section{Related Work}
Robot air hockey has emerged as a valuable benchmark for studying high-speed, contact-rich manipulation. Early systems largely relied on classical control pipelines combining vision-based state estimation, analytic trajectory prediction, and rule-based strategy selection. These approaches are computationally efficient and highly interpretable, enabling robots to react reliably under well-modeled conditions \cite{b1}.

More recent work has pushed these classical ideas toward greater speed and reactivity. Liu et al. demonstrate that deterministic pipelines can still achieve high-performance play by refining state estimation and optimizing intercept strategies \cite{b1}. Their results show that well-designed analytic systems can outperform learning-based methods when timing constraints are tight and model uncertainty is low. Similar challenges arise in other high-speed tasks such as robot table tennis and juggling, where the robot must constantly adjust to unpredictable object motion. Work in these areas highlights how trajectory prediction and precise timing are essential for consistent performance \cite{b4,b6}.

The Robot Air Hockey Challenge has encouraged research into the development of reinforcement learning, imitation learning, and hybrid architectures that attempt to learn broader strategies while retaining classical control for low-level execution\cite{b2}. While these methods can discover more nuanced offensive behavior, they often struggle with precise timing and reliability \cite{b2}.

Our approach builds on these insights by combining classical predictive modeling with simple strategy logic. Our architecture consists of 2D kinematic prediction, fast intercept selection, and real-time trajectory generation for the IIWA arm. Unlike the previously discussed systems focusing solely on defense or specialized striking, our method aims to balance both blocking incoming shots while calculating an offensive path. 

\section{Proposed Approach}

Our group will deliver a reproducible simulation of two KUKA IIWA robots playing air hockey on a regulation table with side walls and goals. We will provide scene assets that handle friction and restore the state of the puck after a goal. The source for these environment assets can be found in the appendix [1]. In addition, we will program a Python stack that demonstrates the IIWA’s ability to shoot, defend against shots, and play against an opponent. A stretch goal would be to implement reinforcement learning as a high-level decision maker to learn policies for offense and defense.

\subsection{Core System}

We will use Python for the overall simulation and trajectories. The culminating environment will include a static table, wall geometry, two IIWA’s mounted at opposite ends, and a lightweight, cylindrical puck moving with low friction. The table will contain guard rails on the midpoint line (spanning across the table’s shorter dimension) so that each IIWA’s cannot expand into their opponent’s domain. The puck’s movement will be realistic and physics-driven. It will bounce off of walls at a speed that depends on the velocity and direction of collision. In interactions between puck and paddle, the puck will generate an escape velocity proportional to the amount of force received by the paddle.

The core of our system involves control planning and trajectory predictions (Fig. 1). We will compute puck trajectories using 2D kinematics to predict when/where the puck will cross the table half or goal mouth. On defense, our IIWA will pick the earliest feasible intercept that prevents a goal. On offense, our IIWA will pick a direct or wall shot based on opponent coverage. Slight randomization will allow for shots to be unpredictable. It will be necessary to compute an end-effector trajectory so each IIWA can arrive at a puck-paddle intercept with a desired striking velocity or blocking pose. After one of these actions, the IIWA will recover to an equilibrium pose near the center of the table to prepare for its next move.

\begin{figure}[t]
    \centering
\includegraphics[width=0.8\linewidth]{air_hockey_sections.png}
    \caption{The table contains three zones: green, blue, and red. In the red zone, the IIWA’s goal intersection states will be farther from the initial puck state at the computation time. This effect lessens if the puck is entering the blue or green zones. We hypothesize that this approach will help balance the unpredictability of puck ricochets off the wall.}
    \label{fig:placeholder}
\end{figure}

\subsection{Interception Strategy}

It will be interesting to uncover how our IIWA’s prediction patterns influence the effectiveness of their strikes. For example, if the puck is bouncing off of a wall on the left side of the table, would it be more effective for our IIWA to intercept the puck farther from its original state? For humans, wall shots are more unpredictable, so we tend to position our paddle defensively (closer to the goal) rather than extending our paddle to meet the puck. This strategy is the opposite for a puck traveling down the center of the table. Our initial idea, described in Figure 2, was to separate the table into zones that determine the intersection goal states of the IIWA’s.


\section{Timeline}

\begin{table}[h]
\caption{Timeline}
\centering
\begin{tabular}{c p{2.2cm} p{2.2cm} p{2.2cm}}
\hline
\textbf{Week} & \textbf{Tasks} & \textbf{Outcome} & \textbf{Assigned Member} \\
\hline
Oct. 26th - 31st & Review feedback from initial project proposals & Finalized project proposal & Ryan, Guillermo, Tye \\
Nov. 2nd - 8th &  & First Project Check-in Completed &  \\
Nov. 9th - 13th & Implement robotics kinematics for pose-to-target movement. Verify Joint trajectories for positioning against the puck & & Guillermo, Tye \\ Nov. 16th - 21st & Work on object detection and physics model prediction for the puck. & & Ryan, Guillermo \\ 
Nov. 23rd - 29th & Verify interception of end effector and the puck. &  & Ryan, Guillermo, Ty \\ Nov. 30th - Dec. 6th & Fine tune full integration, ensuring all systems still work as expected after initial integration. & Finalize Paper and Video. Final Project check-in. & Ryan, Guillermo, Tye \\
Dec. 7th - 13th & Presentation Practice Sessions  & Presentation & Ryan, Tye\\
\hline
\end{tabular}
\label{tab:mytable}
\end{table}


% Keep your text and graphic files separate until after the text has been 
% formatted and styled. Do not number text heads---{\LaTeX} will do that 
% for you.

% \subsection{Abbreviations and Acronyms}\label{AA}
% Define abbreviations and acronyms the first time they are used in the text, 
% even after they have been defined in the abstract. Abbreviations such as 
% IEEE, SI, MKS, CGS, ac, dc, and rms do not have to be defined. Do not use 
% abbreviations in the title or heads unless they are unavoidable.

% \subsection{Units}
% \begin{itemize}
% \item Use either SI (MKS) or CGS as primary units. (SI units are encouraged.) English units may be used as secondary units (in parentheses). An exception would be the use of English units as identifiers in trade, such as ``3.5-inch disk drive''.
% \item Avoid combining SI and CGS units, such as current in amperes and magnetic field in oersteds. This often leads to confusion because equations do not balance dimensionally. If you must use mixed units, clearly state the units for each quantity that you use in an equation.
% \item Do not mix complete spellings and abbreviations of units: ``Wb/m\textsuperscript{2}'' or ``webers per square meter'', not ``webers/m\textsuperscript{2}''. Spell out units when they appear in text: ``. . . a few henries'', not ``. . . a few H''.
% \item Use a zero before decimal points: ``0.25'', not ``.25''. Use ``cm\textsuperscript{3}'', not ``cc''.)
% \end{itemize}

% \subsection{Equations}
% Number equations consecutively. To make your 
% equations more compact, you may use the solidus (~/~), the exp function, or 
% appropriate exponents. Italicize Roman symbols for quantities and variables, 
% but not Greek symbols. Use a long dash rather than a hyphen for a minus 
% sign. Punctuate equations with commas or periods when they are part of a 
% sentence, as in:
% \begin{equation}
% a+b=\gamma\label{eq}
% \end{equation}

% Be sure that the 
% symbols in your equation have been defined before or immediately following 
% the equation. Use ``\eqref{eq}'', not ``Eq.~\eqref{eq}'' or ``equation \eqref{eq}'', except at 
% the beginning of a sentence: ``Equation \eqref{eq} is . . .''

% \subsection{\LaTeX-Specific Advice}

% Please use ``soft'' (e.g., \verb|\eqref{Eq}|) cross references instead

% of ``hard'' references (e.g., \verb|(1)|). That will make it possible
% to combine sections, add equations, or change the order of figures or
% citations without having to go through the file line by line.

% Please don't use the \verb|{eqnarray}| equation environment. Use
% \verb|{align}| or \verb|{IEEEeqnarray}| instead. The \verb|{eqnarray}|
% environment leaves unsightly spaces around relation symbols.

% Please note that the \verb|{subequations}| environment in {\LaTeX}
% will increment the main equation counter even when there are no
% equation numbers displayed. If you forget that, you might write an
% article in which the equation numbers skip from (17) to (20), causing
% the copy editors to wonder if you've discovered a new method of
% counting.

% {\BibTeX} does not work by magic. It doesn't get the bibliographic
% data from thin air but from .bib files. If you use {\BibTeX} to produce a
% bibliography you must send the .bib files. 

% {\LaTeX} can't read your mind. If you assign the same label to a
% subsubsection and a table, you might find that Table I has been cross
% referenced as Table IV-B3. 

% {\LaTeX} does not have precognitive abilities. If you put a
% \verb|\label| command before the command that updates the counter it's
% supposed to be using, the label will pick up the last counter to be
% cross referenced instead. In particular, a \verb|\label| command
% should not go before the caption of a figure or a table.

% Do not use \verb|\nonumber| inside the \verb|{array}| environment. It
% will not stop equation numbers inside \verb|{array}| (there won't be
% any anyway) and it might stop a wanted equation number in the
% surrounding equation.

% \subsection{Some Common Mistakes}\label{SCM}
% \begin{itemize}
% \item The word ``data'' is plural, not singular.
% \item The subscript for the permeability of vacuum $\mu_{0}$, and other common scientific constants, is zero with subscript formatting, not a lowercase letter ``o''.
% \item In American English, commas, semicolons, periods, question and exclamation marks are located within quotation marks only when a complete thought or name is cited, such as a title or full quotation. When quotation marks are used, instead of a bold or italic typeface, to highlight a word or phrase, punctuation should appear outside of the quotation marks. A parenthetical phrase or statement at the end of a sentence is punctuated outside of the closing parenthesis (like this). (A parenthetical sentence is punctuated within the parentheses.)
% \item A graph within a graph is an ``inset'', not an ``insert''. The word alternatively is preferred to the word ``alternately'' (unless you really mean something that alternates).
% \item Do not use the word ``essentially'' to mean ``approximately'' or ``effectively''.
% \item In your paper title, if the words ``that uses'' can accurately replace the word ``using'', capitalize the ``u''; if not, keep using lower-cased.
% \item Be aware of the different meanings of the homophones ``affect'' and ``effect'', ``complement'' and ``compliment'', ``discreet'' and ``discrete'', ``principal'' and ``principle''.
% \item Do not confuse ``imply'' and ``infer''.
% \item The prefix ``non'' is not a word; it should be joined to the word it modifies, usually without a hyphen.
% \item There is no period after the ``et'' in the Latin abbreviation ``et al.''.
% \item The abbreviation ``i.e.'' means ``that is'', and the abbreviation ``e.g.'' means ``for example''.
% \end{itemize}
% An excellent style manual for science writers is \cite{b7}.

% \subsection{Authors and Affiliations}\label{AAA}
% \textbf{The class file is designed for, but not limited to, six authors.} A 
% minimum of one author is required for all conference articles. Author names 
% should be listed starting from left to right and then moving down to the 
% next line. This is the author sequence that will be used in future citations 
% and by indexing services. Names should not be listed in columns nor group by 
% affiliation. Please keep your affiliations as succinct as possible (for 
% example, do not differentiate among departments of the same organization).

% \subsection{Identify the Headings}\label{ITH}
% Headings, or heads, are organizational devices that guide the reader through 
% your paper. There are two types: component heads and text heads.

% Component heads identify the different components of your paper and are not 
% topically subordinate to each other. Examples include Acknowledgments and 
% References and, for these, the correct style to use is ``Heading 5''. Use 
% ``figure caption'' for your Figure captions, and ``table head'' for your 
% table title. Run-in heads, such as ``Abstract'', will require you to apply a 
% style (in this case, italic) in addition to the style provided by the drop 
% down menu to differentiate the head from the text.

% Text heads organize the topics on a relational, hierarchical basis. For 
% example, the paper title is the primary text head because all subsequent 
% material relates and elaborates on this one topic. If there are two or more 
% sub-topics, the next level head (uppercase Roman numerals) should be used 
% and, conversely, if there are not at least two sub-topics, then no subheads 
% should be introduced.

% \subsection{Figures and Tables}\label{FAT}
% \paragraph{Positioning Figures and Tables} Place figures and tables at the top and 
% bottom of columns. Avoid placing them in the middle of columns. Large 
% figures and tables may span across both columns. Figure captions should be 
% below the figures; table heads should appear above the tables. Insert 
% figures and tables after they are cited in the text. Use the abbreviation 
% ``Fig.~\ref{fig}'', even at the beginning of a sentence.

% \begin{table}[htbp]
% \caption{Table Type Styles}
% \begin{center}
% \begin{tabular}{|c|c|c|c|}
% \hline
% \textbf{Table}&\multicolumn{3}{|c|}{\textbf{Table Column Head}} \\
% \cline{2-4} 
% \textbf{Head} & \textbf{\textit{Table column subhead}}& \textbf{\textit{Subhead}}& \textbf{\textit{Subhead}} \\
% \hline
% copy& More table copy$^{\mathrm{a}}$& &  \\
% \hline
% \multicolumn{4}{l}{$^{\mathrm{a}}$Sample of a Table footnote.}
% \end{tabular}
% \label{tab1}
% \end{center}
% \end{table}

% \begin{figure}[htbp]
% \centerline{\includegraphics{fig1.png}}
% \caption{Example of a figure caption.}
% \label{fig}
% \end{figure}

% Figure Labels: Use 8 point Times New Roman for Figure labels. Use words 
% rather than symbols or abbreviations when writing Figure axis labels to 

% avoid confusing the reader. As an example, write the quantity 
% ``Magnetization'', or ``Magnetization, M'', not just ``M''. If including 
% units in the label, present them within parentheses. Do not label axes only 
% with units. In the example, write ``Magnetization (A/m)'' or ``Magnetization 
% \{A[m(1)]\}'', not just ``A/m''. Do not label axes with a ratio of 
% quantities and units. For example, write ``Temperature (K)'', not 
% ``Temperature/K''.

% \section*{Acknowledgment}

% The preferred spelling of the word ``acknowledgment'' in America is without 
% an ``e'' after the ``g''. Avoid the stilted expression ``one of us (R. B. 
% G.) thanks $\ldots$''. Instead, try ``R. B. G. thanks$\ldots$''. Put sponsor 
% acknowledgments in the unnumbered footnote on the first page.

\begin{thebibliography}{00}
\bibitem{b1} P. Liu, D. Tateo, H. Bou-Ammar and J. Peters, "Efficient and Reactive Planning for High Speed Robot Air Hockey," 2021 IEEE/RSJ International Conference on Intelligent Robots and Systems (IROS), Prague, Czech Republic, 2021, pp. 586-593, doi: 10.1109/IROS51168.2021.9636263.
\bibitem{b2} P. Liu et al., “A Retrospective on the Robot Air Hockey Challenge: Benchmarking Robust, Reliable, and Safe Learning Techniques for Real-world Robotics,” arXiv.org, 2024. https://arxiv.org/abs/2411.0571
\bibitem{b3} R. L. Bishop and M. W. Spong, “Control of a robotic air hockey system,” in Proc. IEEE Int. Conf. Robot. Autom. (ICRA), Leuven, Belgium, May 1998, pp. 289–294.
\bibitem{b4} M. Bühler, D. E. Koditschek, and P. J. Kindlmann, “A family of robot juggling tasks,” Int. J. Robot. Res., vol. 13, no. 2, pp. 101–118, Apr. 1994.
\bibitem{b5} M. Diehl, H. J. Ferreau, and N. Haverbeke, “Efficient numerical methods for nonlinear MPC and moving horizon estimation,” in Nonlinear Model Predictive Control, L. Magni, D. M. Raimondo, and F. Allgöwer, Eds. Berlin, Heidelberg: Springer, 2009, pp. 391–417.
\bibitem{b6} K. Mülling, J. Kober, and J. Peters, “Learning to select and generalize movements in robot table tennis,” IEEE Trans. Robot., vol. 29, no. 4, pp. 934–946, Aug. 2013.
\bibitem{b7} B. Siciliano, L. Sciavicco, L. Villani, and G. Oriolo, Robotics: Modelling, Planning and Control. London, UK: Springer, 2009.
\end{thebibliography}

\end{document}
